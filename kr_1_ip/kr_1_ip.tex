% arara: xelatex

\documentclass[12pt]{article} % размер шрифта

\usepackage{etex} % extend
\usepackage{tikz} % картинки в tikz
\usepackage{microtype} % свешивание пунктуации

\usepackage{diagbox}
\usepackage{slashbox}
\usepackage{tabularx}
\usepackage{comment}

\usepackage{tikzlings}
\usepackage{tikzducks}

\usepackage{array} % для столбцов фиксированной ширины
\usepackage{verbatim} % для вставки комментариев

\usepackage{indentfirst} % отступ в первом параграфе

\usepackage{sectsty} % для центрирования названий частей

\allsectionsfont{\centering} % приказываем центрировать все sections

\usepackage{amsmath,  amsfonts} % куча стандартных математических плюшек

\usepackage[top=1.5cm,  left=1.5cm,  right=1.5cm,  bottom=1.5cm]{geometry} % размер текста на странице

\usepackage{lastpage} % чтобы узнать номер последней страницы

\usepackage{enumitem} % дополнительные плюшки для списков
%  например \begin{enumerate}[resume] позволяет продолжить нумерацию в новом списке
\usepackage{caption} % подписи к картинкам без плавающего окружения figure

\usepackage{comment} % длинные комментарии

\usepackage{fancyhdr} % весёлые колонтитулы
\pagestyle{fancy}
\lhead{Теория вероятностей и статистика,  НИУ-ВШЭ}
\chead{}
\cfoot{}
\rfoot{}
\renewcommand{\headrulewidth}{0.4pt}
\renewcommand{\footrulewidth}{0.4pt}

\usepackage{todonotes} % для вставки в документ заметок о том,  что осталось сделать
% \todo{Здесь надо коэффициенты исправить}
% \missingfigure{Здесь будет картина Последний день Помпеи}
% команда \listoftodos — печатает все поставленные \todo'шки

\usepackage{booktabs} % красивые таблицы
% заповеди из документации:
% 1. Не используйте вертикальные линии
% 2. Не используйте двойные линии
% 3. Единицы измерения помещайте в шапку таблицы
% 4. Не сокращайте .1 вместо 0.1
% 5. Повторяющееся значение повторяйте,  а не говорите "то же"

\usepackage{fontspec} % поддержка разных шрифтов
\usepackage{polyglossia} % поддержка разных языков

\setmainlanguage{russian}
\setotherlanguages{english}

\setmainfont{Linux Libertine O} % выбираем шрифт

% можно также попробовать Helvetica,  Arial,  Cambria и т.Д.

% чтобы использовать шрифт Linux Libertine на личном компе, 
% его надо предварительно скачать по ссылке
% http://www.linuxlibertine.org/index.php?id=91&L=1

\newfontfamily{\cyrillicfonttt}{Linux Libertine O}
% пояснение зачем нужно шаманство с \newfontfamily
% http://tex.stackexchange.com/questions/91507/

\AddEnumerateCounter{\asbuk}{\russian@alph}{щ} % для списков с русскими буквами
\setlist[enumerate,  2]{label=\asbuk*), ref=\asbuk*} % списки уровня 2 будут буквами а) б) \ldots 

%% эконометрические и вероятностные сокращения
\DeclareMathOperator{\Cov}{Cov}
\DeclareMathOperator{\Corr}{Corr}
\DeclareMathOperator{\Var}{Var}
\DeclareMathOperator{\E}{\mathbb{E}}
\DeclareMathOperator{\D}{Var}
\newcommand \hb{\hat{\beta}}
\newcommand \hs{\hat{\sigma}}
\newcommand \htheta{\hat{\theta}}
\newcommand \s{\sigma}
\newcommand \hy{\hat{y}}
\newcommand \hY{\hat{Y}}
\newcommand \e{\varepsilon}
\newcommand \he{\hat{\e}}
\newcommand \hVar{\widehat{\Var}}
\newcommand \hCorr{\widehat{\Corr}}
\newcommand \hCov{\widehat{\Cov}}
\newcommand \cN{\mathcal{N}}
\newcommand{\R}{\mathbb{R}}

\let\P\relax
\DeclareMathOperator{\P}{\mathbb{P}}

%\fbox{
%  \begin{minipage}{24em}
%    Фамилия,  имя и номер группы (печатными буквами):\vspace*{3ex}\par
%    \noindent\dotfill\vspace{2mm}
%  \end{minipage}
%  \begin{tabular}{@{}l p{0.8cm} p{0.8cm} p{0.8cm} p{0.8cm} p{0.8cm}@{}}
% %\toprule
% Задача & 1 & 2 & 3 & 4 & 5\\ 
% \midrule
% Балл  &  &  & & & \\
% \midrule
% %\bottomrule
% \end{tabular}
% }    


\begin{document}

\fbox{
  \begin{minipage}{24em}
    Фамилия, имя, отчество (печатными буквами):\vspace*{3ex}\par
    \noindent\dotfill\vspace{2mm} \\
     Фамилия семинариста:\vspace*{3ex}\par
    \noindent\dotfill\vspace{2mm}
  \end{minipage}
  \begin{tabular}{@{}l p{0.8cm} p{0.8cm} p{0.8cm} p{0.8cm} p{0.8cm}@{}}
%\toprule
Задача & 1 & 2 & 3 & 4 & 5\\ 
\midrule
 &  &  & & &\\
Балл  &  &  & & & \\
\midrule
%\bottomrule
\end{tabular}
}    

\newpage
\text{ }
\newpage
\text{ }
\newpage
\text{ }
\newpage

\lfoot{15 октября 2022 года, 102 года со дня рождения Марио Пьюзо}
\rhead{вариант $\aleph_0$}
\rfoot{\thepage/2}


\setcounter{page}{1}

\begin{minipage}{0.6\textwidth}
\begin{quote}
    Оставь пистолет. Захвати пирожные. 
\end{quote}
\begin{flushright}
    \textit{Марио Пьюзо, Крёстный отец}
\end{flushright}
\end{minipage}


\begin{enumerate}
\item Вито и Дженко открыли компанию по импорту оливкового масла. 
Они получили $X$ литров масла и разливают его поровну по $N+1$ вместительным бутылкам. 
Величины $X$ и $N$ независимы, $X$ имеет экспоненциальное распределение с ожиданием 10 литров,
а $N$ — пуассоновское распределение с ожиданием $9$ бутылок. 

\begin{enumerate}
    % \item Найдите вероятность в каждой бутылке будет ровно литр масла. 
    \item {[5]} Найдите вероятность того, что в каждой бутылке будет меньше либо равно литра масла,
    если число бутылок $N + 1$ равно трем.
    \item {[5]} Найдите вероятность того, что в каждой бутылке будет меньше либо равно литра масла, 
    если число бутылок $N + 1$ оказалось не больше трех. 
    \item {[5]} Найдите $F(1)$, если $F$ — функция распределения количества масла в каждой бутылке.
\end{enumerate}


\item На праздник ровно $5\%$ жён итальянских мафиози получили в подарок цветы. 
Цветы получают в подарок только от мужа. 
Также известно, что $0.5\%$ жён получили в подарок пирожные,
причём половина жён получила их от мужа, а половина — от брата.
Среди жён, получивших пирожные от мужа, $90\%$ получили в подарок цветы. 
Среди жён, получивших пирожные от брата, $5\%$ получили в подарок цветы.

\begin{enumerate}
    \item {[5]} Кармела получила в подарок цветы. 
    Какова условная вероятность того, что она получила пирожные в подарок от мужа?
    \item {[10]} Талия получила в подарок цветы и пирожные. 
    Какова условная вероятность того, что она получила пирожные в подарок от мужа?
\end{enumerate}


\item В особняке дона Вито Корлеоне собрались в круг $n>2$ гостей. 
У двоих из гостей есть по теннисному мячу. 
Одновременно и независимо друг от друга эти двое бросают свои мячи случайно выбираемым гостям. 
Если мячи были брошены одному гостю, то он объявляется преемником Крёстного отца. 
Если мячи были брошены разным гостям, то новые раунды бросков продолжаются по тем же правилам. 
Обозначим буквой $X$ количество раундов, которое потребуется для определения преемника. 

\begin{enumerate}
    \item {[5]} Найдите вероятность $\P(X = 2)$. 
    \item {[5]} Найдите функцию распределения величины $Y = \min\{X, 3\}$.
    \item {[5]} За сколько раундов в среднем будет определён преемник?
\end{enumerate}

\newpage

\item Количество глав семей, $X$, собранных доном Корлеоне для переговоров, имеет производящую вероятности функцию 
\[
g(t) = 0.2 t + 0.3 t^2 + 0.5 t^3.    
\]
\begin{enumerate}
    \item {[5]} Постройте график функции распределения величины $X$. 
    \item {[5]} Найдите $g'(1)$ и объясните, какой смысл имеет величина $g'(1)$ для произвольной функции производящей вероятности.
\end{enumerate}

\item Чтобы уйти от преследования, консильери хочет сесть в первый попавшийся автобус. 
На остановку приходят автобусы трёх маршрутов, $A$, $B$ и $C$. 
Интервалы между автобусами каждого маршрута равны ровно пяти минутам. 
Автобусы разных маршрутов приходят независимо друг от друга. 
Обозначим буквой $T$ время ожидания консильери на остановке в минутах.

\begin{enumerate}
    \item {[5]} Найдите вероятность $\P(T \leq 1)$.
    \item {[5]} Найдите функцию плотности и функцию распределения $T$.
    \item {[5]} Найдите ожидание $\E(T)$.
\end{enumerate}

% \item Консильери подкидывает правильную монетку до тех пор, пока не выпадет $HTH$. 
% Найдите ожидаемое число подбрасываний. 


% \item {[бонусный вопрос]} Какой день какого месяца был вчера в Италии в 1582 году?
% 4 октября, даты с 5 по 14 октября отсутствовали из-за введения папой Григорием нового календаря

\end{enumerate}


\end{document}