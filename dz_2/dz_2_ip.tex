% arara: xelatex

\documentclass[12pt]{article} % размер шрифта

\usepackage{etex} % extend
\usepackage{tikz} % картинки в tikz
\usepackage{microtype} % свешивание пунктуации

\usepackage{diagbox}
\usepackage{slashbox}
\usepackage{tabularx}
\usepackage{comment}

\usepackage{tikzlings}
\usepackage{tikzducks}

\usepackage{array} % для столбцов фиксированной ширины
\usepackage{verbatim} % для вставки комментариев

\usepackage{indentfirst} % отступ в первом параграфе

\usepackage{sectsty} % для центрирования названий частей

\allsectionsfont{\centering} % приказываем центрировать все sections

\usepackage{amsmath,  amsfonts} % куча стандартных математических плюшек

\usepackage[top=1.5cm,  left=1.5cm,  right=1.5cm,  bottom=1.5cm]{geometry} % размер текста на странице

\usepackage{lastpage} % чтобы узнать номер последней страницы

\usepackage{enumitem} % дополнительные плюшки для списков
%  например \begin{enumerate}[resume] позволяет продолжить нумерацию в новом списке
\usepackage{caption} % подписи к картинкам без плавающего окружения figure

\usepackage{comment} % длинные комментарии

\usepackage{fancyhdr} % весёлые колонтитулы
\pagestyle{fancy}
\lhead{Теория вероятностей и статистика,  НИУ-ВШЭ}
\rhead{Домашняя работа II}
\chead{}
\cfoot{}
\rfoot{}
\renewcommand{\headrulewidth}{0.4pt}
\renewcommand{\footrulewidth}{0.4pt}

\usepackage{physics}

\usepackage{url}

\usepackage{todonotes} % для вставки в документ заметок о том,  что осталось сделать
% \todo{Здесь надо коэффициенты исправить}
% \missingfigure{Здесь будет картина Последний день Помпеи}
% команда \listoftodos — печатает все поставленные \todo'шки

\usepackage{booktabs} % красивые таблицы
% заповеди из документации:
% 1. Не используйте вертикальные линии
% 2. Не используйте двойные линии
% 3. Единицы измерения помещайте в шапку таблицы
% 4. Не сокращайте .1 вместо 0.1
% 5. Повторяющееся значение повторяйте,  а не говорите "то же"

\usepackage{fontspec} % поддержка разных шрифтов
\usepackage{polyglossia} % поддержка разных языков

\setmainlanguage{russian}
\setotherlanguages{english}

\setmainfont{Linux Libertine O} % выбираем шрифт

% можно также попробовать Helvetica,  Arial,  Cambria и т.Д.

% чтобы использовать шрифт Linux Libertine на личном компе, 
% его надо предварительно скачать по ссылке
% http://www.linuxlibertine.org/index.php?id=91&L=1

\newfontfamily{\cyrillicfonttt}{Linux Libertine O}
% пояснение зачем нужно шаманство с \newfontfamily
% http://tex.stackexchange.com/questions/91507/

\AddEnumerateCounter{\asbuk}{\russian@alph}{щ} % для списков с русскими буквами
\setlist[enumerate,  2]{label=\asbuk*), ref=\asbuk*} % списки уровня 2 будут буквами а) б) \ldots 

%% эконометрические и вероятностные сокращения
\DeclareMathOperator{\Cov}{Cov}
\DeclareMathOperator{\Corr}{Corr}
\DeclareMathOperator{\Var}{Var}
\DeclareMathOperator{\E}{\mathbb{E}}
\DeclareMathOperator{\D}{Var}
\newcommand \hb{\hat{\beta}}
\newcommand \hs{\hat{\sigma}}
\newcommand \htheta{\hat{\theta}}
\newcommand \s{\sigma}
\newcommand \hy{\hat{y}}
\newcommand \hY{\hat{Y}}
\newcommand \e{\varepsilon}
\newcommand \he{\hat{\e}}
\newcommand \hVar{\widehat{\Var}}
\newcommand \hCorr{\widehat{\Corr}}
\newcommand \hCov{\widehat{\Cov}}
\newcommand \cN{\mathcal{N}}
\newcommand{\R}{\mathbb{R}}



\let\P\relax
\DeclareMathOperator{\P}{\mathbb{P}}

%\fbox{
%  \begin{minipage}{24em}
%    Фамилия,  имя и номер группы (печатными буквами):\vspace*{3ex}\par
%    \noindent\dotfill\vspace{2mm}
%  \end{minipage}
%  \begin{tabular}{@{}l p{0.8cm} p{0.8cm} p{0.8cm} p{0.8cm} p{0.8cm}@{}}
% %\toprule
% Задача & 1 & 2 & 3 & 4 & 5\\ 
% \midrule
% Балл  &  &  & & & \\
% \midrule
% %\bottomrule
% \end{tabular}
% }    


% делаем короче интервал в списках
\setlength{\itemsep}{0pt}
\setlength{\parskip}{0pt}
\setlength{\parsep}{0pt}

\begin{document}

Во всех задачах можно использовать python/R/julia/C++ или любой другой язык программирования с открытыми исходниками. 
Вполне могут возникать уравнения или оптимизационные задачи, которые решаются только численно.
Если при решении задачи использовался код, то его нужно привести. 
Можно использовать chatgpt с указанием полного диалога в решении.  
Решение нужно прислать в виде \textbf{одного pdf файла}. 
Про бутстрэп и перестановочные тесты можно и нужно почитать \url{https://arxiv.org/abs/1411.5279}.
При симуляциях всегда добивайтесь воспроизводимости, для питона стоит глянуть \url{https://builtin.com/data-science/numpy-random-seed}.
Для задач 4, 5 и 6 потребуются данные по результатам экзамена по теории вероятностей этого года 
по 30-балльной шкале. Их можно найти на вики-страничке курса, \url{http://wiki.cs.hse.ru/Econ_probability_2022-23}. 

Жесткий дедлайн: 21:00 12 июня, 
при сдаче работы до 21:00 11 июня начисляется бонус +10\%.  

\begin{enumerate}

    \item Однажды в Самарканде турист заказывал Яндекс-такси. 
    На десятом заказе впервые приехал таксист, который уже раньше приезжал к туристу. 
    Для упрощения предположим, что все $n$ таксистов Самарканда всегда на работе и приезжают равновероятно.

    \begin{enumerate}
      \item {[5]} Постройте график функции правдоподобия как функции от общего количества такси $n$. 
      Найдите оценку числа $n$ методом максимального правдоподобия. 
      \item {[5]} Постройте график математического ожидания номера заказа, 
      на котором происходит первый повторный приезда, как функции от общего количества такси $n$. 
      Найдите оценку числа $n$ методом моментов.
      \item {[15]} Предположим, что настоящее $n$ равно 100. 
      Проведя 10000 симуляций вызовов такси до первого повторного, рассчитайте 10000 оценок метода моментов и 10000 оценок методам максимального правдоподобия. 
      Постройте гистограммы для оценок двух методов. 
      Оцените смещение, дисперсию и среднеквадратичную ошибку двух методов. 
    \end{enumerate}

    \item Однажды в Самарканде турист заказывал Яндекс-такси. 
    На десятом заказе он обнаружил, что у таксистов было 6 разных имён.
    Для упрощения предположим, что все $n$ имён среди таксистов встречаются равновероятно и независимо от 
    поездки к поездке. 

    \begin{enumerate}
      \item {[5]} Постройте график функции правдоподобия как функции от общего количества имён $n$. 
      Найдите оценку числа $n$ методом максимального правдоподобия. 
      \item {[5]} Постройте график математического ожидания числа разных имён у 10 таксистов, 
      как функции от общего количества имён $n$. 
      Найдите оценку числа $n$ методом моментов.
      \item {[15]} Предположим, что настоящее $n$ равно 20. 
      Проведя 10000 симуляций десяти вызовов такси, рассчитайте 10000 оценок метода моментов и 10000 оценок методам максимального правдоподобия. 
      Постройте гистограммы для оценок двух методов. 
      Оцените смещение, дисперсию и среднеквадратичную ошибку двух методов. 
    \end{enumerate}

    
    
    \item Иноагент Иннокентий по 20 наблюдениям строит 95\%-й доверительный интервал для математического ожидания
    несколькими способами: классический асимптотический нормальный интервал, 
    с помощью наивного бутстрэпа, с помощью бутстрэпа $t$-статистики.

    
    \begin{enumerate}
      \item {[15]} Для каждого способа с помощью 10000 симуляций оцените 
      вероятность того, что номинально 95\%-й доверительный интервал фактически накрывает математическое ожидание,
      если наблюдения распределены экспоненциально с интенсивностью 1. 
      \item {[5]} Пересчитайте вероятности накрытия, если наблюдения имеют распределение Стьюдента с тремя степенями свободы.
      \item {[5]} Какой способ оказался лучше?
    \end{enumerate}
  

    \item Проверьте гипотезу о том, что ожидаемые результаты экзамена по теории вероятностей тех, у кого фамилия начинается с гласной буквы и с согласной буквы, равны.
    В качестве альтернативной гипотезы возьмите гипотезу о неравенстве. 

    \begin{enumerate}
      \item {[5]} Используйте тест Уэлча. 
      \item {[5]} Используйте наивный бутстрэп. 
      \item {[5]} Используйте бутстрэп $t$-статистики. 
      \item {[5]} Используйте перестановочный тест.
    \end{enumerate}
    В каждом случае укажите $P$-значение и статистический вывод для уровня значимости $5\%$.

  \item Составьте таблицу сопряжённости, поделив студентов писавших экзамен на четыре группы по двум признакам:
  набрал ли больше медианы или нет, на согласную или гласную букву начинается фамилия. 

  \begin{enumerate}
    \item {[5]} Постройте 95\% асимптотический интервал для отношения шансов хорошо написать экзамен («несогласных» к «согласным»).
    Проверьте гипотезу о том, что отношение шансов равно 1 и укажите P-значение.
    \item {[5]} Постройте 95\% асимптотический интервал для отношения вероятностей хорошо написать экзамен. 
    Проверьте гипотезу о том, что отношение вероятностей равно 1 и укажите P-значение.
    %\item {[5]} Постройте 95\% асимптотический интервал для разности вероятностей хорошо написать экзамен.  
    % Проверьте гипотезу о том, что разность вероятностей равна 0 и укажите P-значение.
    \item {[5]} Постройте 95\% интервал для отношения шансов хорошо написать экзамен 
    с помощью наивного бутстрэпа. Проверьте гипотезу о том, что отношение шансов равно 1 и укажите P-значение.
  \end{enumerate}
  В качестве альтернативной гипотезы используйте гипотезу о неравенстве. 



    \item Иноагент Иннокентий Вероятностно-Статистический считает, что длина фамилии положительно влияет 
    на результат экзамена по теории вероятностей. 
    А именно, он предполагает, что ожидаемый результат за экзамен прямо пропорционален длине фамилии,
    $\E(Y_i) = \beta F_i$, где $Y_i$ — результат за экзамен по 30-балльной шкале, 
    $F_i$ — количество букв в фамилии. 

    \begin{enumerate}
      \item {[10]} Оцените $\beta$ методом моментов. 
      \item {[5]} С помощью перестановочного теста найдите $P$-значение и формально протестируйте гипотезу о том, что $\beta = 0$.
    \end{enumerate}


    \item {[10]} С помощью chatgpt решите любую задачу из нашего курса теории вероятностей и статистики. 
    Можно брать задачи из прошлых контрольных, лекций, семинаров и даже этого домашнего задания.
    В качестве ответа приведите полный диалог с chatgpt. 
 
    Простой диалог в виде двух реплик условия и ответа chatgpt даёт 6 баллов. 
    Сложный диалог с наводками, указанием chatgpt на ошибки и их исправлением — 10 баллов. 
 
    Пример инструкции как зарегаться, \url{https://journal.tinkoff.ru/chatgpt-in-russia/}.
 


    \item {[5]} Укажите любой источник по теории вероятностей или статистике, который вам оказался полезен 
    в течение года. 
    Это может быть статья, видео, задача, всё что угодно. Объясните, с чем конкретно этот источник помог 
    разобраться. Лучше привести в пример внешний источник, не упомянутый на вики курса,
    но можно и внутренний. 
    

\end{enumerate}


\end{document}