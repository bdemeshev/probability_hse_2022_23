% arara: xelatex

\documentclass[12pt]{article} % размер шрифта

\usepackage{etex} % extend
\usepackage{tikz} % картинки в tikz
\usepackage{microtype} % свешивание пунктуации

\usepackage{diagbox}
\usepackage{slashbox}
\usepackage{tabularx}
\usepackage{comment}

\usepackage{tikzlings}
\usepackage{tikzducks}

\usepackage{array} % для столбцов фиксированной ширины
\usepackage{verbatim} % для вставки комментариев

\usepackage{indentfirst} % отступ в первом параграфе

\usepackage{sectsty} % для центрирования названий частей

\allsectionsfont{\centering} % приказываем центрировать все sections

\usepackage{amsmath,  amsfonts} % куча стандартных математических плюшек

\usepackage[top=1.5cm,  left=1.5cm,  right=1.5cm,  bottom=1.5cm]{geometry} % размер текста на странице

\usepackage{lastpage} % чтобы узнать номер последней страницы

\usepackage{enumitem} % дополнительные плюшки для списков
%  например \begin{enumerate}[resume] позволяет продолжить нумерацию в новом списке
\usepackage{caption} % подписи к картинкам без плавающего окружения figure

\usepackage{comment} % длинные комментарии

\usepackage{fancyhdr} % весёлые колонтитулы
\pagestyle{fancy}
\lhead{Теория вероятностей и статистика,  НИУ-ВШЭ}
\chead{}
\cfoot{}
\rfoot{}
\renewcommand{\headrulewidth}{0.4pt}
\renewcommand{\footrulewidth}{0.4pt}

\usepackage{todonotes} % для вставки в документ заметок о том,  что осталось сделать
% \todo{Здесь надо коэффициенты исправить}
% \missingfigure{Здесь будет картина Последний день Помпеи}
% команда \listoftodos — печатает все поставленные \todo'шки

\usepackage{booktabs} % красивые таблицы
% заповеди из документации:
% 1. Не используйте вертикальные линии
% 2. Не используйте двойные линии
% 3. Единицы измерения помещайте в шапку таблицы
% 4. Не сокращайте .1 вместо 0.1
% 5. Повторяющееся значение повторяйте,  а не говорите "то же"

\usepackage{fontspec} % поддержка разных шрифтов
\usepackage{polyglossia} % поддержка разных языков

\setmainlanguage{russian}
\setotherlanguages{english}

\setmainfont{Linux Libertine O} % выбираем шрифт

% можно также попробовать Helvetica,  Arial,  Cambria и т.Д.

% чтобы использовать шрифт Linux Libertine на личном компе, 
% его надо предварительно скачать по ссылке
% http://www.linuxlibertine.org/index.php?id=91&L=1

\newfontfamily{\cyrillicfonttt}{Linux Libertine O}
% пояснение зачем нужно шаманство с \newfontfamily
% http://tex.stackexchange.com/questions/91507/

\AddEnumerateCounter{\asbuk}{\russian@alph}{щ} % для списков с русскими буквами
\setlist[enumerate,  2]{label=\asbuk*), ref=\asbuk*} % списки уровня 2 будут буквами а) б) \ldots 

%% эконометрические и вероятностные сокращения
\DeclareMathOperator{\Cov}{Cov}
\DeclareMathOperator{\Corr}{Corr}
\DeclareMathOperator{\Var}{Var}
\DeclareMathOperator{\E}{\mathbb{E}}
\DeclareMathOperator{\D}{Var}
\newcommand \hb{\hat{\beta}}
\newcommand \hs{\hat{\sigma}}
\newcommand \htheta{\hat{\theta}}
\newcommand \s{\sigma}
\newcommand \hy{\hat{y}}
\newcommand \hY{\hat{Y}}
\newcommand \e{\varepsilon}
\newcommand \he{\hat{\e}}
\newcommand \hVar{\widehat{\Var}}
\newcommand \hCorr{\widehat{\Corr}}
\newcommand \hCov{\widehat{\Cov}}
\newcommand \cN{\mathcal{N}}
\newcommand{\R}{\mathbb{R}}

\let\P\relax
\DeclareMathOperator{\P}{\mathbb{P}}

%\fbox{
%  \begin{minipage}{24em}
%    Фамилия,  имя и номер группы (печатными буквами):\vspace*{3ex}\par
%    \noindent\dotfill\vspace{2mm}
%  \end{minipage}
%  \begin{tabular}{@{}l p{0.8cm} p{0.8cm} p{0.8cm} p{0.8cm} p{0.8cm}@{}}
% %\toprule
% Задача & 1 & 2 & 3 & 4 & 5\\ 
% \midrule
% Балл  &  &  & & & \\
% \midrule
% %\bottomrule
% \end{tabular}
% }    


\begin{document}

\fbox{
  \begin{minipage}{24em}
    Фамилия, имя, отчество (печатными буквами):\vspace*{3ex}\par
    \noindent\dotfill\vspace{2mm} \\
     Фамилия семинариста:\vspace*{3ex}\par
    \noindent\dotfill\vspace{2mm}
  \end{minipage}
  \begin{tabular}{@{}l p{0.8cm} p{0.8cm} p{0.8cm} p{0.8cm} p{0.8cm}@{}}
%\toprule
Задача & 1 & 2 & 3 & 4 & 5\\ 
\midrule
 &  &  & & &\\
Балл  &  &  & & & \\
\midrule
%\bottomrule
\end{tabular}
}    

\newpage
\text{ }
\newpage
\text{ }
\newpage
\text{ }
\newpage

\lfoot{190 лет со дня выхода первого издания «Евгения Онегина»}
\rfoot{vale}
\rhead{20 марта 2023 года}


\setcounter{page}{1}

\begin{minipage}{0.6\textwidth}
\begin{quote}
    И Ленский пешкою ладью \\
    Берет в рассеянье свою. 
\end{quote}
\begin{flushright}
    \textit{Александр Пушкин, Евгений Онегин}
\end{flushright}
\end{minipage}


\begin{enumerate}


\item В течении 10 ночей снится чудный сон Татьяне. Ей снится будто бы она идёт по снеговой поляне, 
печальной мглой окружена, а из сугроба является большой взъерошенный медведь. 
Размеры медведей $X_i$ независимы и имеют функцию плотности 
\[
f(x) = \begin{cases}    
(\theta + x)/(\theta + 1.5), \text{ если } x \in [1;2], \\
0,  \text{ иначе}.
\end{cases}
\]
Помогите Татьяне оценить интенсивность хандры суженого $\theta$ с помощью метода моментов. 

\item Служанки со всего двора используют метод максимального правдоподобия для 
гадания про своих барышень. 
Количества мурчаний кота в очередной день марта, $X_i$ —
независимые случайные величины с распределением заданным формулой $\P(X_i = k) = (1-p)^{k-1}p$,
где $p$ — вероятность того, что муж окажется богатым. 

Оцените вероятность того, что муж окажется богатым, если за март было 10 дней, 
в которые кот мурлыкнул ровно 1 раз, 15 дней, в которые кот мурлыкнул ровно 2 раза,
и 5 дней, в которые кот мурлыкнул \textit{не менее} 3-х раз. 

\item Какое минимальное количество робертов нужно сыграть игроку, чтобы 
с вероятностью не менее 80\% быть уверенным, что доля выигранных партий в выборке 
отличается от его истинной вероятности выиграть не более, чем на $0.01$?

\item Евгений каждый свой вечер проводит у Лариных $X_i$ минут. 
Величины $X_i$ независимы и одинаково распределены. 

Евгений использует необычную оценку 
\[
\hat\mu = \frac{\sum_{i=1}^n X_i}{n+1}
\]
для неизвестного параметра $\mu = \E(X_i)$. 

\begin{enumerate}
    \item Является ли эта оценка несмещённой? состоятельной?
    \item При каком условии на $\mu$ и $\sigma^2 = \Var(X_i)$ эта
    необычная оценка будет иметь среднеквадратичную ошибку $MSE = \E((\hat \mu - \mu)^2)$ меньше, чем у классической $\hat\mu=\bar X$?
\end{enumerate}



\item Евгений опросил жителей Москвы и Санкт-Петербурга,
какой из трёх видов отдыха они предпочитают: прогулки, чтенье, сон глубокий. 
Из 300 опрошенных москвичей прогулки предпочитают 100 человек, чтенье — 50 и сон глубокий — 150 человек.
Из 200 опрошенных санкт-петербуржцев прогулки предпочитают 80 человек, чтение — 70 и сон глубокий — 50 человек. 

\begin{enumerate}
    \item Постройте 95\% асимптотический доверительный интервал для доли москвичей, предпочитающих прогулки. 
    \item Постройте 95\% асимптотический доверительный интервал для разницы  доли москвичей и доли петербуржцев, предпочитающих прогулки.
    \item (*) Постройте 95\% асимптотический доверительный интервал для разницы  доли москвичей, предпочитающих прогулки, и 
    доли москвичей, предпочитающих сон глубокий. 
\end{enumerate}


\end{enumerate}


\end{document}